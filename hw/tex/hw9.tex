\documentclass[12pt, leqno]{article} %% use to set typesize
\include{common}

\begin{document}

\hdr{2019-11-18}{2019-11-25}

You may (and should) talk about problems with each other and with me,
providing attribution for any good ideas you might get.  Your final
write-up should be your own.

% Look at convergence of Gauss-Seidel and comment (smoothing)
% Do a subspace solve and comment (coarse solve)
% Combined iteration and comment

For this assignment, we will consider the one-dimensional model
system on $[0,1]$ with a grid of $n = 100$ interior points.
The interior points have coordinates $x_j = jh$ for $h = 1/(n+1)$.
You should complete the code framework in the class repository
as outlined below (Julia or MATLAB/Octave) and comment on what
you see.

\paragraph*{1: Gauss-Seidel sweeps}
Implement a simple block Gauss-Seidel sweep in the {\tt solve\_bgs}
function.  What do you observe about the plots of the discrete sine
transform of the error?

\paragraph*{2: Bubnov-Galerkin}
Implement a simple Bubnov-Galerkin approximation to the correction
equation
\[
  T (\Delta u) = h^2 f - T \hat{u}
\]
in the function {\tt solve\_coarse}.  The tester uses a space of
polynomials (represented in the Chebyshev basis for stability);
how does the error vary depending on the polynomial degree?
Also compute the optimal error in the space and the quasi-optimality
constant.  How tight is the standard quasi-optimality bound in
this case?

\paragraph*{3: Two-grid iteration}
The final loop in the tester alternates between Galerkin projections
with a polynomial space and Gauss-Seidel sweeps.  Discuss why
the combined iteration works better than either approach alone.


\end{document}
