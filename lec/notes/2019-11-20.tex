\documentclass[12pt, leqno]{article} %% use to set typesize
\include{common}

\begin{document}

\hdr{2019-11-20}

%% \section{Model reduction}

%% Our focus in this section is methods for solving a single linear
%% system at a time.  Often, we want to solve many closely-related
%% linear systems with different matrices.  As a simple example,
%% we might want to evaluate
%% \[
%%   (A-\sigma I) x(\sigma) = b
%% \]
%% for several different values of $\sigma$ within some range; more
%% generally, we might want to solve linear systems $A(s) x(s) = b(s)$
%% where $A$ and $b$ depend smoothly on some low-dimensional parameter
%% vector $s$ that varies over a bounded set.  In such settings, one often
%% finds (and can sometimes prove via interpolation theory) that $x(s)$
%% lies close to a space $\calV$ that can be computed. For example, we
%% might find that an adequate space $\calV$ spanned by sample solutions
%% $x(s_1), x(s_2), \ldots$; we could then choose a corresponding trial
%% space $\calW$ as the basis for a Galerkin scheme.  Hence, we may
%% estimate $x(\sigma)$ very quickly (online) after a more expensive
%% computation to construct a basis for an appropriate approximation space
%% (offline).

%% There are a wide-variety of techniques that employ this general idea.
%% These include model reduction methods from control theory
%% (moment-matching methods that use Krylov subspaces, truncated balanced
%% realization methods that involve solving Sylvester equations, etc);
%% global-basis methods for the solution of PDEs (e.g.~the so-called {\em
%% empirical interpolation method}); and many other methods for both
%% linear and nonlinear problems.  While it is not a focus for this course,
%% the approach is so simple and broadly applicable that I would feel bad
%% if you left not knowing about it.

\section{Krylov subspaces}

The {\em Krylov subspace} of dimension $k$ generated by
$A \in \bbR^{n \times n}$ and $b \in \bbR^n$ is
\[
  \mathcal{K}_k(A,b)
    = \operatorname{span}\{ b, Ab, \ldots, A^{k-1} b \}
    = \{ p(A) b : p \in \mathcal{P}_{k-1} \}.
\]
Krylov subspaces are a natural choice for subspace-based methods for
approximate linear solves, for two reasons:
\begin{itemize}
\item If all you are allowed to do with $A$ is compute matrix-vector
  products, and the only vector at hand is $b$, what else would you do?
\item The Krylov subspaces have excellent approximation properties.
\end{itemize}

Krylov subspaces have several properties that are worthy of comment.
Because the vectors $A^{j} b$ are proportional to the vectors obtained
in power iteration, one might reasonably (and correctly)
assume that the space quickly contains good approximations to the
eigenvectors associated with the largest magnitude eigenvalues.
Krylov subspaces are also {\em shift-invariant}, i.e. for any $\sigma$
\[
  \mathcal{K}_k(A-\sigma I, b) = \mathcal{K}_k(A,b).
\]
By choosing different shifts, we can see that the Krylov subspaces
tend to quickly contain not only good approximations to the eigenvector
associated with the largest magnitude eigenvalue, but to all
``extremal'' eigenvalues.

Most arguments about the approximation properties of Krylov subspaces
derive from the characterization of the space as all vectors $p(A) b$
where $p \in \mathcal{P}_{k-1}$ and from the spectral mapping theorem,
which says that if $A = V \Lambda V^{-1}$ then
$p(A) = V p(\Lambda) V^{-1}$.  Hence, the distance between
an arbitrary vector (say $d$) and the Krylov subspace is
\[
  \min_{p \in \mathcal{P}_{k-1}}
  \left\| V \left[ p(\Lambda) V^{-1} b - V^{-1} d \right] \right\|.
\]
As a specific example, suppose that we want to choose $\hat{x}$
in a Krylov subspace in order to minimize the residual $A \hat{x} - b$.
Writing $\hat{x} = p(A) b$, we have that we want to minimize
\[
  \|[A p(A)-I] b\| = \|q(A) b\|
\]
where $q(z)$ is a polynomial of degree at most $k$ such that $q(1) = 1$.
The best possible residual in this case is bounded by
\[
  \|q(A) b\| \leq \kappa(V) \|q(\Lambda)\| \|b\|,
\]
and so the relative residual can be bounded in terms of the condition
number of $V$ and the minimum value that can bound $q$ on the spectrum
of $A$ subject to the constraint that $q(0) = 1$.

\section{Chebyshev polynomials}

Suppose now that $A$ is symmetric positive definite, and we seek to
minimize $\|q(A) b\| \leq \|q(\Lambda)\| \|b\|$.  Controlling $q(z)$
on all the eigenvalues is a pain, but it turns out to be simple to
instead bound $q(z)$ over some interval $[\alpha_1, \alpha_n]$
The polynomial we want is the {\em scaled and shifted Chebyshev polynomial}
\[
  q_m(z) =
  \frac{T_m\left( (z-\bar{\alpha})/\rho \right)}
       {T_m\left( -\bar{\alpha}/\rho \right)}
\]
where $\bar{\alpha} = (\alpha_n + \alpha_1)/2$ and
$\rho = (\alpha_n-\alpha_1)/2$.

The Chebyshev polynomials $T_m$ are defined by the recurrence
\begin{align*}
  T_0(x) &= 1 \\
  T_1(x) &= x \\
  T_{m+1}(x) &= 2x T_m(x) - T_{m-1}(x), \quad m \geq 1.
\end{align*}
The Chebyshev polynomials have a number of remarkable properties, but
perhaps the most relevant in this setting is that
\[
  T_m(x) =
  \begin{cases}
    \cos(m \cos^{-1}(x)), & |x| \leq 1, \\
    \cosh(m \cosh^{-1}(x)), &|x| \geq 1
  \end{cases}.
\]
Thus, $T_m(x)$ oscillates between $\pm 1$ on the interval $[-1,1]$,
and then grows very quickly outside that interval.  In particular,
\[
  T_{m}(1 + \epsilon) \geq \frac{1}{2} (1+m\sqrt{2\epsilon}).
\]
Thus, we have that on $[\alpha_, \alpha_n]$,
$|q_m| \leq \frac{2}{1+m\sqrt{2\epsilon}}$
where
\[
  \epsilon = \bar{\alpha}/\rho-1
  = \frac{2\alpha_1}{\alpha_n-\alpha_1}
  = 2 \left( \kappa(A)-1 \right)^{-1},
\]
and hence
\begin{align*}
  |q_m(z)|
  &\leq \frac{2}{1+2m/\sqrt{\kappa(A)-1}} \\
  &= 2\left( 1- \frac{2m}{\sqrt{\kappa(A)-1}}\right) + O\left(\frac{m^2}{\kappa(A-1)}\right).
\end{align*}
Hence, we expect to reduce the optimal residual in this case
by at least about $2/\sqrt{\kappa(A)-1}$ at each step.

\section{Chebyshev: Uses and Limitations}

We previously sketched out an approach for analyzing the convergence of
methods based on Krylov subspaces:
\begin{enumerate}
\item
  Characterize the Krylov subspace of interest in terms of polynomials,
  i.e. $\mathcal{K}_k(A,b) = \{ p(A)b : p \in \mathcal{P}_{k-1} \}$.
\item
  For $\hat{x} = p(A) b$, write an associated error (or residual)
  in terms of a related polynomial in $A$.
\item
  Phrase the problem of minimizing the error, residual, etc.~in terms
  of minimizing a polynomial $q(z)$ on the spectrum of $A$
  (call this $\Lambda(A)$).  The polynomial $q$ must generally satisfy
  some side constraints that prevent the zero polynomial from being
  a valid solution.
\item
  Let $\Lambda(A) \subset \Omega$, and write
  \[
    \max_{\lambda \in \Lambda(A)} |q(\lambda)| \leq
    \max_{z \in \Omega} |q(z)|.
  \]
  The set $\Omega$ should be simpler to work with than the set of
  eigenvalues.  The simplest case is when $A$ is symmetric positive
  definite and $\Omega = [\lambda_1, \lambda_n]$.
\item
  The optimization problem can usually be phrased in terms of special
  polynomial families.  The simplest case, when $\Omega$ is just an
  interval, usually leads to an analysis via Chebyshev polynomials.
\end{enumerate}
The analysis sketched above is the basis for the convergence analysis
of the Chebyshev semi-iteration, the conjugate gradient method, and
(with various twists) several other Krylov subspace methods.

The advantage of this type of analysis is that it leads to convergence
bounds in terms of some relatively simple property of the matrix, such
as the condition number.  The disadvantage is that the approximation of
the spectral set $\Lambda(A)$ by a bounding region $\Omega$ can lead to
rather pessimistic bounds.  In practice, the extent to which we are able
to find good solutions in a Krylov subspace often depends on the
``clumpiness'' of the eigenvalues.  Unfortunately, this ``clumpiness''
is rather difficult to reason about a priori!  Thus, the right way to evaluate
the convergence of Krylov methods in practice is usually to try them out,
plot the convergence curves, and see what happens.

\end{document}
