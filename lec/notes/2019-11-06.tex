\documentclass[12pt, leqno]{article} %% use to set typesize
\include{common}
\newcommand{\calK}{\mathcal{K}}

\begin{document}

\hdr{2019-11-06}

\section{Algorithms}

There are several flavors of symmetric eigenvalue solvers for which
there is no equivalent (stable) nonsymmetric solver.  We discuss four
algorithmic ideas: the workhorse QR algorithm that we have already
seen, the Jacobi iteration, divide-and-conquer, and bisection with
inverse iteration.

The details of these algorithms are quite technical (particularly
the divide-and-conquer method and bisection with inverse iteration).
But even if you are not planning to focus in numerical linear algebra,
you should know a little about the shape of these algorithms.  Why?
Three reasons:
\begin{itemize}
\item Intellectual fun!
\item To make an informed choice of algorithms.
\item To re-use building blocks.
\end{itemize}

\section{Symmetric QR}

\subsection{The eigenvalues of symmetric $A$}

Like the nonsymmetric QR iteration, the symmetric QR iteration involves
an initial reduction, but to a {\em tridiagonal} form.  This is really
the same as the Hessenberg reduction step; but a symmetric Hessenberg
matrix is tridiagonal, and we can use that.  Similarly, when we perform
bulge-chasing, the intermediate matrices remain symmetric, and so we
never have a matrix that is more than a few elements away from
tridiagonal.  For the symmetric case, there is a little difference in
how we choose shifts: Wilkinson shifts are fine since there are only
real eigenvalues -- no need for the Francis double-shift strategy. But
otherwise, the main difference is that each step of the tridiagonal QR
iteration maps between representations with $O(n)$ parameters in $O(n)$
time.  Each eigenvalue converges in roughly a constant number of
iterations, so the cost to compute all eigenvalues of a tridiagonal is
$O(n^2)$.  Compared to the $O(n^3)$ cost of reducing to tridiagonal form
in the first place, the cost of solving for the eigenvalues of the
tridiagonal is thus quit modest.

If we want all the eigenvalues of a sparse matrix, and only want the
eigenvalues, the algorithm is basically the fastest option. But if we
want eigenvectors as well, then the QR iteration is more expensive,
costing an additional $O(n^3)$; other methods run faster.

\subsection{QR iteration for singular values}

Now consider the case of computing the singular values of a matrix $A$.
We could compute the singular values directly from the eigenvalues of
the Gram matrix $A^T A$ (or $AA^T$, or the Golub-Kahan matrix).  But
the backward error associated with tridiagonal reduction is proportional
to the norm of the matrix $A^T A$ (or the square norm of $A$) and this
can look quite big compared to the square of the smallest singular values.
So rather than work with $A^T A$ {\em explicitly}, we prefer to manipulate
$A$ in order to run the same algorithm {\em implicitly}.

The first step of the QR iteration for the singular value problem is
thus {\em bidiagonalization}; that is, we compute
\[
  A = \hat{U} B \hat{V}^T
\]
where $B$ is an upper bidiagonal matrix.  Note that
\[
  A^T A = \hat{V} B^T B \hat{V}^T = \hat{V} T \hat{V}^T,
\]
i.e.~$B$ is the Cholesky factor of the tridiagonal matrix $A$ that
we would obtain by tridiagonalization of the Gram matrix $A^T A$.
But we can compute $B$ directly by alternately applying
transformations to $A$ from the left and the right.

After the bidiagonal reduction, we want to do implicit QR steps.
As a shift, we use the square root of the trailing
corner element of the tridiagonal $B^T B$; in terms of $B$, this is
just the norm of the last column:
\[
  \sigma = \sqrt{b_{n-1,n}^2 + b_{n,n}^2}.
\]
With this shift in hand, we could apply the first step of shifted
QR and complete the process implicitly via bulge chasing in the same
way we did for the nonsymmetric case.  In practice, there is an alternate
algorithm (the {\em dqds} method) that enjoys extra stability benefits,
allowing us to compute the singular values of $B$ to high relative
accuracy.\footnote{%
Of course, we usually lose high relative accuracy of the small singular
values through the initial reduction to
bidiagonal --- the backward error for that reduction is only small
relative to the norm of $A$.}

\section{Jacobi iteration}

A {\em Jacobi rotation} is a symmetric transformation that
diagonalizes a $2 \times 2$ (sub)matrix:
\[
  J^T A J = \Lambda
\]
In terms of scalars, this means solving
\[
  \begin{bmatrix} c & -s \\ s & c \end{bmatrix}
  \begin{bmatrix} \alpha & \beta \\ \beta & \gamma \end{bmatrix}
  \begin{bmatrix} c & s \\ -s & c \end{bmatrix} =
  \begin{bmatrix} \lambda_1 & 0 \\ 0 & \lambda_2 \end{bmatrix},
\]
where $c = \cos(\theta)$ and $s = \sin(\theta)$.  A numerically
stable method for carrying out this computation is described in
Golub and Van Loan\footnote{Or on Wikipedia!}; we will leave
the details aside for the purposes of these notes.

Given an algorithm for computing Jacobi rotations, the idea
of the Jacobi iteration is to incrementally apply Jacobi rotations
to each $2 \times 2$ principle minors of $A$ in turn.  The code is
remarkably simple:

\lstinputlisting{code/eigen/jacobi_sweep.m}

Each time we apply a Jacobi iteration to rows and columns $k$ and $l$,
we reduce the sum of squares off the main diagonal by $a_{kl}^2$. The
iteration converges quadratically, and typically we can stop after 5--10
sweeps.  Each sweep costs $O(n^3)$, and has cost comparable to the cost
of a tridiagonalization step before running QR iteration.  Hence, Jacobi
iteration is rather slow.  On the other hand, it tends to compute the
small eigenvalues of $A$ much more accurate than competing methods that
start with a tridiagonalization step.

\subsection{Divide and conquer}

The symmetric QR iteration and the Jacobi iteration are methods that
an appropriately motivated student could reasonably code\footnote{%
I do not really recommend this, but it is not implausible.}
The divide and conquer method, on the other hand, is much more
numerically subtle.  Nonetheless, the ideas are quite interesting,
and it is worth spending a moment describing the strategy at a
high level.

Like the symmetric QR iteration, the divide-and-conquer method is
preceded by a tridiagonalization process.  After tridiagonalization,
we think of the tridiagonal matrix as a block $2 \times 2$ matrix
\[
  T =
  \begin{bmatrix}
    T_{11} & \beta e_{k} e_1^T \\
    \beta e_1 e_{k}^T & T_{22}
  \end{bmatrix} =
  \begin{bmatrix}
    \tilde{T}_{11} & 0 \\ 0 & \tilde{T}_{22}
  \end{bmatrix} +
  \beta
  \begin{bmatrix} e_k \\ e_1 \end{bmatrix}
  \begin{bmatrix} e_k \\ e_1 \end{bmatrix}^T.
\]
where $T_{11}$ and $T_{22}$ are tridiagonal submatrices of the original
tridiagonal, and $\tilde{T}_{11}$ and $\tilde{T}_{22}$ are these
submatrices with $\beta$ subtracted from a corner entry.  We compute
the eigendecompositions $Q_{11}^T \tilde{T}_{11} Q_{11} = D_1$ and
similarly for $\tilde{T}_{22}$ by applying the divide-and-conquer
method recursively.  We then have
\[
  Q^T T Q = D + \beta uu^T
\]
where $Q$ is a block diagonal orthogonal matrix with diagonal blocks
$Q_{11}$ and $Q_{22}$, and $u$ consists of the last row of $Q_{11}$
and the first row of $Q_{22}$ stacked atop each other.

Assuming we can perform this reduction, we now need to find
solutions to
\[
  \det(D + \beta uu^T - \lambda I) = 0.
\]
Without loss of generality, consider the case where $u$ has no zero
elements\footnote{%
If there is a zero element in $u$, we have a converged eigenvalue on
the diagonal, and can deflate it away}; then none of the diagonal
entries of $D$ are eigenvalues of $A$, but we can write the eigenvalues
of $A$ as solutions to
\[
  f(\lambda) = \det(D + \beta uu^T - \lambda I) / \det(D - \lambda I)
  = 1 + \beta u^T (D-\lambda I)^{-1} u,
\]
where we have used the identity (a good homework problem)
\[
  \det(I+XY^T) = \det(I+Y^T X).
\]
The equation $f(\lambda) = 0$ is known as a {\em secular equation},
and it is a particularly nice type of rational function.  The values
$d_i$ are poles of $f$, and there is one solution to $f(\lambda) = 0$
between each pair of poles, as well as one that is either smaller
than $\min d_i$ or greater than $\max d_i$.

We can compute solutions to the secular equation very efficiently using
a variant of Newton's method.  Naively, this iteration would seem to
require $O(n)$ time per step in order to evaluate $f$ and its derivatives
at any given point.  However, the evaluation time can be reduced
significantly using the {\em fast multipole method}, so that the overall
time is close to $O(1)$ per step; as a consequence, finding all the
solutions to one secular equation takes $O(n)$ time.

The difficulty in the divide-and-conquer algorithm lies primarily in
obtaining accurate eigenvector estimates.  The problem occurs when
eigenvalues cluster together; in this case, one tends to compute
eigenvector estimates which look good on their own, but which are not
orthogonal to each other.  Fixing this problem while retaining good
performance was a technical tour de force, and we will not even attempt
to do it full justice here.  We suffice it to say that the divide-and-conquer
algorithm is quite fast if we want all eigenvalues and eigenvectors,
{\em particularly} if the eigenvectors are clustered.

\section{Bisection with inverse iteration}

Our final algorithm for the tridiagonal eigenvalue problem\footnote{%
As with QR and divide-conquer, the bisection iteration is typically
preceded by a tridiagonal reduction.
} is based on {\em Sylvester's inertia theorem},
which says that congruent matrices have the same inertia.
In particular, that means that if we perform the symmetric
factorization
\[
  T - \sigma I = L D L^T,
\]
the number of positive, negative, and zero diagonal entries of $D$
is the same as the number of eigenvalues of $A$ that are respectively
greater than, less than, and equal to $\sigma$.  The {\em bisection
algorithm} recursively partitions an initial interval containing all
eigenvalues of $T$ into smaller intervals that either
\begin{itemize}
\item Contain no eigenvalues of $A$;
\item Contain exactly one eigenvalue of $A$; or
\item Are smaller in length than some tolerance.
\end{itemize}
The center point of each interval containing exactly one eigenvalue is
a reasonable estimate for that eigenvalue, good enough to guarantee
convergence of a shift-invert iteration.  And this is the bisection
algorithm in a nutshell.

The main technical difficulty with the bisection algorithm lies not
in the bisection step (which is remarkably well-behaved even though
it involves an unpivoted factorization of an indefinite matrix),
but in the computation of the eigenvectors.  As with the divide-and-conquer
scheme, bisection with inverse iteration tends to yield eigenvector
estimates which individually have small residuals, but which may not be
adequately orthogonal for vectors that correspond to eigenvalues in a
cluster.  One could use explicit re-orthogonalization, but the modern
``Grail'' code (the RRR routines in LAPACK nomenclature) manages
stability in a much more subtle and clever way.
This was the thesis work of Inderjit Dhillon before he turned his
energies to machine learning and data mining; more recently, the SVD
case was the prize-winning thesis work of Paul Willems.  The fact that
the algorithm merits multiple highly-regarded PhD theses should tell
you something of the subtleties in getting it right.

The Grail code is so-called because it has optimal complexity for
computing eigenvectors (given the eigenvalues); to obtain $k$ eigenvectors
requires only $O(kn)$ time.  This is generally the fastest way to
compute a specified subset of eigenvectors, and is often the fastest
way to compute all eigenvectors.

\end{document}
